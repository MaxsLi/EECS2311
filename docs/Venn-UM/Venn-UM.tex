% !TEX encoding = UTF-8
\documentclass[fontsize=12pt,paper=letter,twoside]{scrartcl}
% Various imports and themes
\input{sty/defns.tex} 
\usepackage{pdfpages}
%%%%%%%%%%%%%%%%%%%%%%%%%%%%%%%%%%%%%%%%%%%%%%%%%%%%%%%%%%%%%%%%%%%%%%%%%
\begin{document}
%%%%%%%%%%%%%%%%%%%%%%%%%%%%%%%%%%%%%%%%%%%%%%%%%%%%%%%%%%%%%%%%%%%%%%%%%
% Title and Authors
\newcommand{\mytitle}{
	EECS2311-W20 Project: Venn Create\\ User Manual
}
\ihead[]{\small \mytitle}
\title{\mytitle}
\author{
	Chidalu Agbakwa (216337784) \and
	Jihal Patel (216376436) 	\and
	Shangru Li (214488993) 		\and
	Robert Suwary (215446016)
}
% Display a given date or no date
\date{\today}
\maketitle
%%%%%%%%%%%%%%%%%%%%%%%%%%%%%%%%%%%%%%%%%%%%%%%%%%%%%%%%%%%%%%%%%%%%%%%%%
\subsection*{Revisions}
\begin{tabular}{|l|l|p{3in}|}
	\hline
	Date & Revision& Description \\ 
	\hline
	09 February 2020 
	& 1.0       
	& Initial release of this document\\ 
	\hline
	23 February 2020
	& 1.1
	& Updated for Save/Load feature\\
	\hline
\end{tabular}
\bigskip\bigskip
%%%%%%%%%%%%%%%%%%%%%%%%%%%%%%%%%%%%%%%%%%%%%%%%%%%%%%%%%%%%%%%%%%%%%%%%%
\newpage
\vspace*{2in}
\begin{center}
	\huge{\textbf{User Manual}:\\ Venn Create}
\end{center}
\newpage
%%%%%%%%%%%%%%%%%%%%%%%%%%%%%%%%%%%%%%%%%%%%%%%%%%%%%%%%%%%%%%%%%%%%%%%%%
\newpage
\tableofcontents
\newpage
\listoffigures
%\listoftables
%%%%%%%%%%%%%%%%%%%%%%%%%%%%%%%%%%%%%%%%%%%%%%%%%%%%%%%%%%%%%%%%%%%%%%%%%
\newpage
\section{Introduction}
Welcome to Venn Create - a desktop application that can draw customizable
Venn diagrams.

Venn Create enables user to easily create Venn diagrams with customized
labels, in different size and shape. Which can be used to compare and
contrast two or more objects, events, people, or concepts. Clearly
illustrating the differences and similarities between different entities
and all of this, with ZERO cost.

Venn Create provides a user-friendly interface, so that new users will
be able to use the application with minimum efforts. In addition,
the application provides essential functionalities, such as export/import
existing Venn diagrams, printing and customized theming.

%%%%%%%%%%%%%%%%%%%%%%%%%%%%%%%%%%%%%%%%%%%%%%%%%%%%%%%%%%%%%%%%%%%%%%%%%
\newpage
\section{Get Started}

Venn Create provides an intuitive interface and strive to deliver the best user experience.

\subsection{Introduction Interface}

 After user opens the application, the introduction interface will display:

\begin{figure}[hbt]
	\begin{mdframed}
		\includegraphics[width=\textwidth]{images/intro-screenshot.png}
	\end{mdframed}
	\caption{Introduction Interface}
\end{figure}

On the top left corner there are three options on the menu bar:

\begin{itemize}
	\item[] {
		\textbf{File}: Operation regarding the file, currently, close the application.
	}
	\item[] {
	 	\textbf{Insert}: Add a new circle to the diagram (currently disabled).
	}
	\item[] {
		\textbf{Help}: Open this user manual.
	}
\end{itemize}
	
In the middle, there are two buttons. \textbf{Create New} will allow user
to create a new Venn diagram. \textbf{Get Existing} will enable user to
load an existing Venn diagram to the application.

\subsection{Main Interface}

If user chose to create a new Venn diagrams, the application will switch to the main interface:

\begin{figure}[hbt]
	\begin{mdframed}
		\includegraphics[width=\textwidth]{images/main-screenshot.png}
	\end{mdframed}
	\caption{Main Interface}
\end{figure}

On top, user can customize the title of the current Venn diagram, and the
name of either diagrams.

On the side of the diagrams, user can customize the colour of the diagram
py using a colour picker.

At the bottom left there is a indicator for which side of the diagram the label
is in.

Bottom centre is a text area for putting text for new labels.

Then bottom right, clicking the add button will add the text to the diagram.
And the save button will save the current diagram to a .csv file.

When a user attempts to close the application, the application will also
warn the user to save the progress to a file. In case for future loading.

%%%%%%%%%%%%%%%%%%%%%%%%%%%%%%%%%%%%%%%%%%%%%%%%%%%%%%%%%%%%%%%%%%%%%%%%%

\newpage 
\section{Using the Software}

After getting acquainted with the basic user interface of the software.
User can explore the product and create customized Venn diagram. Here is an example usage:

\textbf{\begin{figure}[hbt]
		\begin{mdframed}
			\includegraphics[width=\textwidth]{images/usage1-screenshot.png}
		\end{mdframed}
		\caption{Example Usage}
\end{figure}}

In this use case, user named the left diagram "Diagram 1" and the right diagram "Diagram 2".

Furthermore, user has already added two text label on the Venn diagrams. Note that, after each text label been added, user can drag it to anywhere on the workplace.

If user wish to add more text label, one can input the content of the new label and click `Add` button. Then drag the new label anywhere needed.

%%%%%%%%%%%%%%%%%%%%%%%%%%%%%%%%%%%%%%%%%%%%%%%%%%%%%%%%%%%%%%%%%
\end{document} 	

